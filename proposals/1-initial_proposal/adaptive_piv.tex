\documentclass{article}

\usepackage{pdfsync}
\usepackage{natbib}

\begin{document}
	\author{Robin Deits\\ rdeits@csail.mit.edu}
	\title{6.375 Proposal: Adaptive PIV}
	\date{\today}
	\maketitle

	\section{Background: PIV} % (fold)
	\label{sec:background}
	Particle Image Velocimetry (PIV) is an optical approach to measuring the flow field of a fluid, and has been used in the study of combustion, water flow, robotics, and many other fields. It involves seeding a fluid with tracking particles and using a laser or other planar lighting system to capture sequential images of the particle positions in a single thin 2D slice of the fluid. By comparing the change in position of groups of particles between the subsequent frames, a measurement of the local flow vector can be computed for each region of the fluid. This process of determining the movement of each section of the image is extremely time-consuming in a sequential programming system, but can be readily parallelized to significantly improve performance \citep{Yu:2006tb}
	% section background (end)

	\section{Adaptive PIV} % (fold)
	\label{sec:adaptive_piv}
	Theunissen et al. proposed a method for improving the performance of PIV in sub-optimial conditions, called Adaptive PIV \citep{Theunissen:2009cr}. Their method uses information about the current density of seeding particles and the prior estimate of the velocity field to update the size and spatial frequency of the sections of the images which are compared to determine local flow. This has the effect of increasing the number of data points in the busiest (highest particle density and highest velocity) parts of the fluid and reducing the number of samples in the most stable areas of the fluid, which can improve the amount of relevant data collected per computational unit. 

	In this project, I will focus on implementing Adaptive PIV on an FPGA to improve computational performance, with the ultimate goal of allowing accurate real-time fluid tracking. I will be expanding on prior work implementing a standard PIV algorithm on an FPGA \citep{Yu:2006tb}. I will also be using a recent \textsc{Matlab} implementation of the Adaptive PIV algorithm by Samvaran Sharma of the Robot Locomotion Group at MIT CSAIL as the reference code for my implementation. 

	The primary benefit of this project should be the parallelization and speedup of the Adaptive PIV algorithm. In order to achieve the desired image size and accuracy, Sharma's current software requires approximately 2.5 seconds per pair of frames, which makes real-time analysis of the fluid flow impossible. In contrast, Yu et al. were able to compute 15 image pairs per second using their FPGA implementation. My goal will be to achieve this result with the added benefits of the adaptive algorithm's focus on the most important areas of the fluid flow.
	% section adaptive_piv (end)



	\bibliographystyle{plain}
	\bibliography{refs}
\end{document}